\documentclass{report}
\usepackage{graphicx}
\usepackage{amsmath}
\usepackage{commath}
\author{Gavin Ridley}
\date{\today}
\title{4.450 Project Report\\ A Realistic, Open-source Frame Optimization Tool: Frame3DD-Opt}

\begin{document}

\section{Abstract}
While a multitude of open-source truss and frame analysis tools can be found online,
the majority of these focus on topology optimization. I designed a new truss optimization
program capable of performing parallel differential evolution on the largest of supercomputers.
The code uses the validated Frame3DD frame analysis to solve the solid mechanics physics.
A new simple technique for treating a global buckling constraint has been developed, and has
been incorporated into the new code. Results are presented for a handful of problems, and
scaling analysis was performed up to XXX cores on the Sawtooth supercomputer based at
Idaho National Laboratory.

\section{Introduction}
The shape of optimal truss structures under academic assumptions is known for a given
topology in both 2D \cite{} and 3D \cite{}. These results, though, rely on a few assumptions
which tend not to hold in real applications. To name a few, these results neglect global
buckling, and penalize tensile stresses in the same way compressive stresses are penalized.
While these optimal structures indeed solve the minimum compliance problem under a given
topology, they do not necessarily solve the minimum mass problem once buckling constraints
are taken into account--including both local and global buckling.

This work seeks to fill this gap and explore the deviations from theoretically optimal geometry
of Michell trusses using a realistic model of truss physics--including both local and global
buckling. This deviation from idealized optimal solutions is easily shown to depend on both
member geometries and mechanical properties. On top of this, these theoretically optimal
solutions tend to assume pinned connections which do not resist moments, but real structures
tend to use connections which do resist moments such as welds and gusset plates.

So, to expound the physical model of trusses I seek to solve, 

\subsection{Literature Review}
Frames or trusses have a tendency to fail in buckling well before ideal
tensile or compressive yield limits are reached--analogous to how a beam
under compression tends to fail in buckling well before it fails due to
compressive yielding. This can easily be shown to be true using Euler's
buckling formula, available from any mechanics of materials text e.g. \cite{gereMechanicsMaterials1996}, and
comparing this buckling stress to that where yielding occurs. For most
geometries relevant to structural members, buckling happens well before
yielding.

A majority of modern truss and frame optimization studies consider the topology
optimization problem introduced by \cite{}, which uses the "gound structure" concept.
At its core, this method simply optimizes a highly connected trusses' member thicknesses
in order to find an optimal truss layout.

\cite{zhaoa} first incisively pointed out the deficiencies of considering only
local buckling in truss optimization problems. This paper points out a few simple
examples where consideration of only local truss buckling 

\end{document}

\documentclass{report}
\usepackage{graphicx}
\usepackage{amsmath}
\usepackage{commath}
\usepackage{biblatex}
\usepackage{algorithm2e}
\usepackage{hyperref}
\addbibresource{/home/gavin/Documents/library.bib}

\author{Gavin Ridley}
\date{\today}
\title{4.450 Project Report\\ Exploration of Nonideality-Induced Deviations from Michell Truss Shapes using Parallel Differential Evolution}

\begin{document}

\section{Abstract}
Discrete volume-optimal truss shapes under a given truss topology have been explored by
several studies, but these tend to restrict analysis to not consider effects such
as local and global buckling in accord with Michell's original analyses.
Because volume-optimized truss shapes under consideration of
buckling depend on both material properties and member geometry, these nonideal cases have
not been explored in the literature. In this work, the truss shape optimization
problem on a given topology is solved for both the idealized case and buckling-constrained case
using an MPI-parallelized implementation of constrained differential evolution using
the open-source Frame3DD code as the physics backend of the optimization. Comparisons
are drawn between each geometry. Source code is available at \url{https://github.com/gridley/Frame3DD-Opt}.

\section{Introduction}

The shape of optimal truss structures under various simplifying assumptions and simple loading patterns
is known for a given topology in both 2D \cite{pragerNoteDiscretizedMichell1974,mazurekGeometricalAspectsOptimum2011} and 3D \cite{jacotStrainTensorMethod2016}.
These geometries are closely related to Michell's continuous optimum truss solution \cite{michellLimitsEconomyMaterial1904}.
These results, though, rely on a few assumptions
which tend not to hold in real applications. To name a few, these results neglect global
buckling, and penalize tensile stresses in the same way compressive stresses are penalized.
While these optimal structures indeed solve the minimum compliance problem under a given
topology, they do not necessarily solve the minimum mass problem once buckling constraints
are taken into account--including both local and global buckling.

While a multitude of open-source truss and frame analysis tools can be found online,
the majority of these focus on topology optimization. I designed a new truss optimization
program capable of performing parallel differential evolution on the largest of supercomputers.
The code uses the validated Frame3DD frame analysis code \cite{gavinUserManualReference}
to solve the solid mechanics physics. Frame3DD can do truss analysis calculations with linear elasticity,
or nonlinear elasticity making consideration of the geometric stiffness matrix. The code calculates whether
a structure buckles under loading by outputting a specific error code when the stiffness
matrix loses the positive-definite property, which indicates that a buckling instability has been reached
given the current load. The code can also model the shear of truss members, but I left that
option off for the analyses presented here. Notably, frame joints are treated as rigid rather than
pinned and thus resist moments, which is different from the typical academic assumption on
trusses of non-moment-resistsing joints. Notably, by taking the approach of checking the stiffness
matrix for semidefiniteness, local buckling is automatically considered in addition to global
buckling since both effects lead to a loss of semi-definiteness. For more information on
elastic stability, the monograph \cite{timoshenkoTheoryElasticStability2009} is recommended.

Frames or trusses have a tendency to fail in buckling well before ideal
tensile or compressive yield limits are reached--analogous to how a beam
under compression tends to fail in buckling well before it fails due to
compressive yielding. This can easily be shown to be true using Euler's
buckling formula, available from any mechanics of materials text e.g. \cite{gereMechanicsMaterials1996}, and
comparing this buckling stress to that where yielding occurs. For most
geometries relevant to structural members, buckling happens well before
yielding.

Multiple prior works have considered optimization of truss structures under both local
and global buckling constraints. Early work on the handling of local buckling constraints
in truss optimization was explored in the two-part study \cite{zhouDCOCOptimalityCriteria1992}.
It was soon realized by the same authors that consideration of only local buckling is a
fundamentally flawed and incorrect approach for truss optimization in the works \cite{zhouDifficultiesTrussTopology1996,rozvanyDifficultiesTrussTopology1996}.
The aforementioned work \cite{zhouDifficultiesTrussTopology1996} is an especially recommended
read due to its brevity and presentation of specific, incisive examples which clearly undermine the
credibility of any truss optimization which only considers local buckling.

To address this, many studies have explored both shape and topology optimization
of trusses under both global and local buckling constraints.
The studies examining buckling-constrained topology optimization, presenting their
own specialized methods, include the member sizing optimization of \cite{ben-talOptimalDesignTrusses2000}
which could easily be applied to a ground structure, as well as \cite{kocvaraModellingSolvingTruss2002}
which solved solved several truss topology optimization problems under a
global stability constraint.

This study does not employ topology optimization by reason of practical considerations.
True, leaving the truss topology severely artificially contracts the design space in which solutions should be found.
While this it is indeed restrictive to specify a topology in advance, the motivation
of this work is aimed towards simply constructible designs as could be done with
lumber from a home improvement store in a backyard. A topology-optimized truss
will usually have a multitude of nodes tangled by members of a variety of thicknesses,
e.g. the results presented in \cite{bendsoeTopologyDesignTruss2004}. In contrast, the geometric complexity
of a \textit{shape-optimized} truss remains approximately constant so long as the topology is
fixed, even if the connecting joint positions have been moved. As such, these structures
remain constructible for small-scale projects.

Multiple other works have optimized trusses using the node movement approach.
\cite{wangTrussShapeOptimization2002} also used evolutionary optimization to modify the
nodal positions of a truss as in this work, but without consideration of buckling.
In constrast to the work \cite{wangTrussShapeOptimization2002}, checks that a local
optimum is indeed obtained after the program terminates via evaluation of the Karush-Kuhn-Tucker condition
is not made, although this could be explored in the future.

Being an evolutionary algorithm, this approach would straightforwardly fit into
the interactive evolutionary interface specification developed in chapter three of \cite{muellerComputationalExplorationStructural2014}.

In this regard, the work \cite{pedersenOptimizationPracticalTrusses2003} also particularly focuses
on practicalities, employing an optimization of frame node positions under physically realistic
constraints including  displacement, stress, characteristic frequencies, and buckling.
The work \cite{pedersenOptimizationPracticalTrusses2003} is likely the study published
thus far most related to this one. While this study does not study constraints on displacement
or vibrational modes, it seeks to further the work of \cite{pedersenOptimizationPracticalTrusses2003}
by exploring the influence of practical constraints on geometries considered to be optimal
in simplified truss physics models.

The analysis method differs from the work \cite{pedersenOptimizationPracticalTrusses2003} in
a few ways though. Firstly, as previously mentioned, no constraints are placed on the characteristic
frequencies of the truss, nor is any consideration made regarding displacements. Only
phenomena which would physically destroy the structure--yielding or buckling--are considered.
Morever, while the study \cite{pedersenOptimizationPracticalTrusses2003} used Danish standard
methods for accounting for engineering uncertainty via safety factor techniques applied to
the elasticity modulus and some other parameters, our analysis simply employs an arbitrary
safety factor applied to either the yield stress, local buckling stress, or global buckling
load.

\section{Methods}

An implementation of the differential evolution algorithm as originally presented by \cite{stornDifferentialEvolutionSimple1997}
has been implemented in Python. As noted in the work \cite{stornDifferentialEvolutionSimple1997}, differential evolution
easily parallelizes, and this has been done via the Message Passing Interface (MPI) \cite{groppUsingMPI2nd1999} bindings
for Python \cite{dalcinParallelDistributedComputing2011}. The outer optimization loop, written in an interpreted scripting
language, wraps a low-level, optimized frame physics solver called Frame3DD, which is available under the GNU license \cite{gavinUserManualReference}.
Because the truss physics is the computationally intensive part of the algorithm, Frame3DD was recompiled using the \texttt{-Ofast}
flag in \texttt{gcc} \cite{stallmanUsingGnuCompiler2009} which does not guarantee standard floating point behavior, but
gives enormous code performance. Solutions obtained with and without this compiler flag were verified to agree. The script
has been named \texttt{Frame3DD-Opt}, and can be found at \url{https://github.com/gridley/Frame3DD-Opt}.

This code has been written to operate in three modes--the first being a linear elasticity problem only without consideration
of buckling as commonly employed in \cite{muellerComputationalExplorationStructural2014} and the MIT course 4.450, and
the second making consideration of local buckling. The third also considers a global buckling constraint.
These three modes are summarized by Table \ref{tab:modes}.
In all modes, only the node positions are the variables
to optimize on, rather than having member thickness as an additional variable as in \cite{pedersenOptimizationPracticalTrusses2003}.
This choice was made in order to reduce the dimensionality of the problem.

\begin{table}
  \caption{Available modes in Frame3DD-Opt}
  \centering
  \begin{tabular}{c|c}
    Mode & Functionality \\
    \hline
    1 & No buckling, minimizes $\sum_i |F_i| L_i$ \\
    2 & Local buckling, minimizes frame mass \\
    3 & Local and global buckling, minimizes frame mass \\
  \end{tabular}
\end{table}

For the mode not considering buckling, the frame solution is essentially just a statics solver, so the member
thicknesses negligibly affect the solution. Joints do resist moments in Frame3DD though, so this statement is
only approximately true. In this mode, the quantity $\sum_i |F_i| L_i$ is the objective variable, which
is known to be equivalent to minimizing frame mass under simple assumptions which neglect buckling \cite{muellerComputationalExplorationStructural2014}.

In the second mode, local buckling is accounted for via the Euler buckling formula. In this code, since the
joints resist moments, I have verified that the buckling effect is well-predicted as the case where each end
of the member is clamped against both rotation and translational movement. To review this formula which can
be found in \cite{gereMechanicsMaterials1996}, the buckling stress is simply:

\begin{equation}
  \sigma = \frac{\pi^2 E I}{(K L)^2}
\end{equation}

$E$ is the modulus of elasticity, $I$ the minimal area moment of inertia perpendicular to the loading axis,
and $L$ the member length.
Theoretically, given the previously mentioned conditions, $K$ should be taken to be 0.5. However, for design
purposes, this is instead commonly taken to be 0.65 to account for defects in materials and other engineering
uncertainties. The same convention is adopted here.

Given a set of truss node coordinates, to evaluate the objective function in one of the latter two
nodes, a linear elasticity calculation is first performed using large member thicknesses. After this,
the members are sized using a safety factor applied to either the tensile yield stress limit or
the same safety factor and the Euler buckling formula for respectively tensioned and compressed members.
After sizing the members using a linear elasticity approach, in mode two, the frame mass is calculated
and this is returned as the objective function. In mode three, an additional nonlinear elasticity calculation
is performed. If the nonlinear stiffness matrix is not positive definite, the design is marked as infeasible
since it will buckle either locally or globally. If the nonlinear stiffness matrix is positive definite,
the design is feasible and the mass of the frame is calculated and this is returned as the objective
function value. For visualization of this procedure, see Algorithms \ref{alg:1}, \ref{alg:2}, \ref{alg:3}
which respectively cover the objective function for each mode.

\begin{algorithm}
  \caption{Mode 1 objective function calculation}
\end{algorithm}

\begin{algorithm}
  \caption{Mode 2 objective function calculation}
\end{algorithm}

\begin{algorithm}
  \caption{Mode 3 objective function calculation}
\end{algorithm}

Sizing members to be resistant against local buckling requires knowledge of the moment of area
of a given member. However, the area of the member is dependent on the result of the linear
elasticity calculation. As such, some invariant geometric factors must additionally be
fixed in order to make the problem well-defined. The code currently has formulas built
in for either circular or square tubes. For the former, the ratio of inner to outer radius
is taken to be fixed and is a user-controlled parameter. For the latter, the ratio of the
wall thickness to the width of the member is fixed as well. The formulas for the moments
of area as a function of area are then calculation by: TODO.

The population size in differential evolution should scale with the number of
variables, if possible. The rule of thumb proposed by \cite{piotrowskiReviewDifferentialEvolution2017}

\section{Conclusions}
Future work
\begin{itemize}
  \item Checking of KKT conditions after run
  \item "Polishing" of solution via gradient-based method after termination of differential evolution
  \item Multiple loading conditions
  \item Other differential evolution selection rules
\end{itemize}

\printbibliography

\end{document}
